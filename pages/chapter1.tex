% Chapter 1

\chapter{引言}

\section{概述}
  计算机在渲染一个平面形状时,一般需要首先将形状的曲边近似转化为折线,由此将待渲染的形状转为多边形,再将其切分为三角形并依次渲染。其中将多边形分割为三角形的步骤是业界所遇到的经典问题,设计效率更高和兼容性更好的算法可以提高切分的速度、准确性以及普适性。
  
  本文实现了一种算法,可在不添加新顶点的情况下快速将任意形状带孔凹多边形划分成数个互不重合的三角形,能正确处理多种复杂情况,且产生的划分规整,便于后续处理。
\section{研究现状}


\section{理论基础}
当下对多边形三角化的研究较为成熟,存在较多的算法实现。其中Ear Clipping(耳切法)算法较为简易,可以处理任意形状的简单多边形,但缺点是可能会产生一些较为细长的三角形。其原理为不断在多边形内寻找可被切除的连续三个顶点构成的三角形。(当一个三角形内部不包含其他顶点且该三角形被原多边形覆盖时我们认为该三角形可切除。)且易知对于任意简单多边形,至少存在两个三角形满足上述条件。该算法的复杂度为\(O(n^2)\)。

Delaunay triangulation[2]算法则会生成一组相对更加优秀的解,其满足形成的三角形中最小角尽可能大。但该算法只适用于凸多边形的划分。虽然计算得出的解在处理后也可用于凹多边形划分,但该划分结果可能会产生新的顶点,导致最终的三角形个数增加。Delaunay三角划分有数种实现方式,其中较为常用的算法如Bowyer-Watson算法,其原理是维护一个合法的Delaunay三角划分,每次向点集中加入新点,并修改新增点附近的三角划分规则。其复杂度为\(O(n^2)\)。或者也可以采用分治法[3],将点集划分为两部分并合并生成的Delaunay三角划分,其时间复杂度为\(O(n*log⁡_2n)\)。

\section{代码结构}


\section{预期效果}


% 与Word等所见即所得编辑工具不同,使用 \LaTeX 工具排版可以将写作与排版过程分离,写作者只需要关心文字的部分,而剩下的排版工作全部交给工具自动完成。

% \section{格式要求}
% 正文宋体小四,正文行间距固定为23磅。

% 通过空一行(两次回车)实现段落换行,仅仅是回车并不会产生新的段落。 \par

% 也可以通过 \verb|\par| 命令来新起一段。

% \section{各节一级标题}
% 我是内容

% \subsection{各节二级标题}
% 你是内容

% \subsubsection{各节三级标题}
% 他是内容

% \section{字体字号}
% {\songti \bfseries 宋体加粗} {\textbf{English}}

% {\songti \itshape 宋体斜体} {\textit{English}}

% {\songti \bfseries \itshape 宋体粗斜体} {\textbf{\textit{English}}}

% \section{编译}
% 本模板必须使用XeLaTeX + BibTeX编译,否则会直接报错。 本模板支持多个平台,结合sublime/vscode/overleaf都可以使用。