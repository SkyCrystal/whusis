% Chapter 5
% \chapter{将非简单多边形标准化}
\chapter{总结与展望}
\section{本文工作总结}
对任意平面多边形计算近似于Delaunay三角网的切分结果是计算机图形学中重要的问题。
本文提出了两种平面多边形切分算法,可以较好的完成对平面中任意多边形的三角剖分,生成贴近于Delaunay三角网的切分方案。
具体而言,本文的主要工作如下。

对于复杂多边形的简化,我们按顶点连边极角序遍历的方式减少了所需的计算量,并采用合适的数据结构以快速排查各边界间的重叠,由此能够更快的计算出多边形中的未标注交点,并将原多边形初步切分为数个简化后的多边形。

在多边形的简单化中,常见算法大多采用贪心算法,任选一可见顶点用于构建桥边。借助单调栈的算法能够在可接受的时间内计算出单个顶点的最优切割边,能一定程度的优化分割结构。

在凹多边形的分割中,当下常常采用耳切法处理多边形的凹节点,有较大的概率产生大量细长三角形。通过启发式算法,我们可以对各个待选边进行评估,并选择少量的切割边,在不影响最终结果的情况下将原多边形分割为凸多边形碎片。

在计算带约束Delaunay三角网的过程中,通过“交换”操作调整三角网结构,避免了重新计算子图Delaunay三角网的递归操作,提高了算法的效率。

\section{未来工作展望}
由于时间和能力限制,本文的算法仍然存在一些待改进的缺陷与不足。
在逐步细分算法中,部分步骤未能贴近理论复杂度下限,存在重复计算,未完整利用已有信息的问题。当前采用的选边方式能够得出结果,但运行速度仍可提升。
在基于带约束Delaunay三角网进行三角划分的过程中,关于交换操作的理论复杂度限制未进行严谨证明,仅通过实验确认了多数情况下其复杂度满足应用需求。
未来可以设计一种更优秀的算法,选择更好的顺序向三角网中加入约束边,以减少插入过程中冲突边重复交换的次数,同时以线性复杂度实现全部冲突边的交换操作,以正确顺序避免交换中“死锁”特殊情况的产生。
% 通常情况下,我们处理的多边形为边界互不相交的简单多边形。对于边界产生了交叉的复杂多边形,需要先进行分割处理,使其变为等价的数个简单多边形。

% 在判断复杂多边形所覆盖范围时,有着多种可行的定义方法。此处我们采用的定义为:对于一个点,从该点向多边形外无限远处连接任意一条射线,当且仅当该射线与多边形边界相交奇数次时点在多边形内部。

% 如图\ref*{complex}所示,该复杂多边形由六个顶点组成,所包含的区域用浅蓝色表示。

% \begin{figure}[htp]
%     \centering
%     \includegraphics[width=0.5\textwidth]
%     {figures/complex.png}
%     \caption{复杂多边形的示例}
%     \label{complex}
%   \end{figure}

% 由于多边形的边界为一条闭合折线,易知对于每个多边形边界上的顶点,与其连接的边为偶数条。因此,射线的位置对相交次数的奇偶性不产生影响。

% 明确了划分规则后,我们可以采用逐层分离的方法将原多边形分割为简单多边形的组合。具体方法如下:

% \begin{enumerate}
%     \item 首先,对于多边形的边界,找出所有未被记录的交叉点,并将对应相交的边分别从交叉点处分割为两部分。
%     \item 选择当前点集中x坐标最小的点,作为新多边形的起始点。
%     \item 遍历当前点的所有连边,按极角序选择当前点到上一个点连线方向顺时针连接到的第一条边\footnote{对于第一个点从x轴负方向开始},将另一个端点加入当前多边形。
%     \item 转到3 直至返回初始点。
%     \item 记录当前多边形,并删除当前多边形对应的边和连边数小于2的顶点。
%     \item 转到2 直至点集为空。
%     \item 对于切割出的每个多边形,计算其内部到多边形外跨越的边数,判断其是否为孔洞。
% \end{enumerate}

% 该算法的本质是记录所有未被顶点标记的交叉点,并重新计算原多边形的边界,判断多边形中每一条边具体所属的多边形边界曲线。

% 经过以上处理,我们完成了非简单多边形的标准化。自此,我们可以计算出任意多边形的三角剖分。
