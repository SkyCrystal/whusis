% 中英文摘要

\begin{cnabstract}{计算几何;平面多边形;三角化}
  平面多边形的三角剖分是计算几何的基础问题之一。计算机在渲染一个几何形状时,常常需要把原图形分解为多个三角形的并,分别对每个三角形依次渲染。通过减少三角剖分中生成三角形的数目,避免细长三角形的产生,可以优化三角剖分所产生的方案,并改善后续渲染的结果。

  本文设计并实现了两种基于平面多边形构建其三角剖分的算法,可在不添加新顶点的情况下快速将任意形状带孔洞多边形划分成数个互不重合的三角形,能正确处理多种复杂情况,且产生的划分中细长三角形较少,能够减小后续操作的误差,便于后续处理。第一种算法从任意平面多边形开始,依次消除了其中的相交边界,内部孔洞,凹点,最终将其转化为数个凸多边形的并集,并分别求出其最优三角剖分。第二种算法则是记录多边形顶点集合,并计算出全部顶点的三角剖分结果,然后在维持三角剖分性质不被破坏的情况下将原多边形的所有边插入剖分结果之中,并删除多边形外部的额外连边,以产生三角剖分结果。
  与当下常用的几种算法相比,本文提到的算法具有更广阔的适用范围,可以正确处理平面中带复杂形状孔洞的任意多边形,而许多其他算法可能会错误的生成部分覆盖多边形孔洞的划分。本文中算法一具有高度模块化的特点,逐步分割的每一步都可以单独取出进行进一步的调优,可以较为透明的查看分割中途的状态,便于后续改进和演示。本文中第二种算法则是基于多边形点集的最优解基础上进行调整得来的结果,在调整后整体性质优秀,执行效率高。

  经过实验,本文的算法可以在数秒内完成对包含十万个顶点的复杂平面多边形的三角划分,且产生的划分结果优秀,三角形形状规整。
  % 计算机在渲染一个形状时,需要首先将形状的曲边转化为折线,由此将待渲染的形状转为多边形,再将其切分为三角形并依次渲染。对于其中将多边形分割为三角形的步骤,设计效率更高和兼容性更好的算法可以提高切分的速度、准确性以及普适性。
  
  % 本文设计并实现了两种基于平面多边形构建其三角剖分的算法,可在不添加新顶点的情况下快速将任意形状带孔洞多边形划分成数个互不重合的三角形,能正确处理多种复杂情况,且产生的划分中细长三角形较少,能够减小后续操作的误差,便于后续处理。
\end{cnabstract}


% \begin{enabstract}{Key1; Key2; Key3}
%   Please use English semicolon and space to separate key words.

%   This is abstract. This is abstract. This is abstract. This is abstract. This is abstract. This is abstract. This is abstract. This is abstract.

%   This is abstract. This is abstract. This is abstract. This is abstract. This is abstract. This is abstract. This is abstract. This is abstract. This is abstract. This is abstract. This is abstract. 
% \end{enabstract}
